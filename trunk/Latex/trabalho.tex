\documentclass[12pt]{article}

\usepackage{dftsl2}
\usepackage{amssymb}
\usepackage{amsmath}
\numberwithin{equation}{section}
\usepackage{enumerate}

\usepackage{graphicx,url}

%\usepackage[brazil]{babel}   
\usepackage[latin1]{inputenc}  

     
\sloppy

\title{Sistemas Lineares II}

\author{Igor Macedo Quintanilha\inst{1}, Roberto de Moura Estev�o Filho\inst{1}}


\address{Departamento de Eng. Elet. e de Comp. - Universidade Federal do Rio de Janeiro\\
  Rio de Janeiro - RJ - Brazil
  \email{\{igormq, robertomest\}@poli.ufrj.br}
}

\begin{document} 

\maketitle

\begin{resumo} 
Como relacionar a \textbf{DFT} de um conjunto de N amostras de um sinal com a transformada de Fourier do sinal?
\end{resumo}


\section{Nota��o}
\begin{itemize}
	\item Geral:
	\begin{itemize}
	\item $x(t)$ � o sinal cont�nuo
		\item $X(j\omega)$ � a \textbf{FT}
		\item $\omega(t) = 
			\begin{cases}
			    1,& -\frac{T_{0}}{2}\leq t \leq\frac{T_{0}}{2}\\0,& \mbox{caso contr�rio} 
		 	\end{cases}$
	\end{itemize}
	\item Janela de Observa��o:
	\begin{itemize}
		\item $W(j\omega)$ � a \textbf{FT} janela (sinc)
		\item $y(t) = x(t)w(t)$ � a por��o de interesse
		\item $Y(j\omega) = \mathcal{F}\{y(t)\}$
		\item $p(t) = \sum_{l=-\infty}^{\infty}\delta(t-lh)$ � o trem de impulsos
		\item $y^\star = y(t)p(t)$ � o trem de impulsos modulados
		\item $y[n]$ s�o as amostras correspondentes
	\end{itemize}
\end{itemize}


\section{Passo 1 - Multiplica��o pela janela}
\begin{equation}
Y(j\omega) = X(j\omega) \ast W(j\omega)
\end{equation}
\section{Passo 2 - Amostragem}
\begin{equation}
Y^\ast(j\omega) = \frac{1}{h} \sum_{k=-\infty}^{\infty} y(j\omega - jk\omega_0),\quad \omega_0 = \frac{2\pi}{h}
\end{equation}
\section{Passo 3 - DTFT de y[n] e a transformada de Fourier de $y[n]$}
\begin{equation}
DTFT\{y[n]\} = FT\{Y^\ast(j\omega)\}|_{\omega = \frac{\Omega}{h}}, \quad t=nh
\end{equation}
\section{Passo 4 - DFT do y[n] com a DTFT de y[n]}
\begin{equation}
\left.DFT\{y[n]\} = DTFT\{y[n]\}\right|_{\Omega = \frac{2\pi}{N}k}
\end{equation}
\section{Passo 5}
\begin{center}
Relacionar a $DFT\{y[n]\}$ com $X(jw)$
\end{center}
\section{Hip�teses}
\begin{itemize}
\item A janela � muito grande (infinita)
\item O sinal $x(t)$ possui banda limitada
\item O sinal $x(t)$ foi amostrado a uma taxa suficientemente alta para evitar aliasing
\end{itemize}
\begin{equation}
DFT\{y[n]\} = \left.\frac{1}{h}\right|_{\omega = \frac{n}{Nh}}
\end{equation}
\section{Nova nota��o}
\begin{itemize}
\item $h$ � o per�odo de amostragem
\item $N$ � o n�mero de pontos da janela
\item $T_0 = (N-1)h$ � a janela de observa��o (tamanho)
\item $N\to\infty$: a resolu��o (frequ�ncia) � aproximadamente $\frac{1}{T_0}$
\end{itemize}
\section{Tarefa}
\begin{equation}
x(t) = 
	\begin{cases}
\frac{A}{a}t + A,& -a\leq t \leq 0\\
-\frac{A}{a}t + A,& 0\leq t \leq a\\
0,& \mbox{caso contr�rio} 
 	\end{cases}
\end{equation}
\begin{itemize}
\item $A = 1$
\item $a = 10^{-3}$
\end{itemize}
\begin{enumerate}[(a)]
\item Mostre que:
\begin{equation}
X(f) = \frac{2A}{a(2\pi f)^2}(1 - cos(2\pi fa))
\end{equation}
\item Crie um script que implementa $X(f)$
\item Desloque circularmente a parte negativa para usar o comando \textbf{ \textit{fft()}} do matlab
\item Utilize a propriedade do deslocamento circular da DFT para determinar a DFT do sinal deslocado\\
\textbf{OBS.:} Propriedade do deslocamento:
Seja $x_0$ o sinal e $X_{0shift}$ o sinal deslocado de m amostras circularmente:
\begin{equation}
DFT\{X_0shift[n]\} = DFT\{x_0[n]\}e^{j\frac{2\pi mk}{N}}
\end{equation}
\item Desloque circularmente a DFT no sentido reverso para obter a DFT do sinal original
\item Defina um vetor de frequ�ncias "adequado"
\item Construir gr�ficos de:
\begin{itemize}
\item Sinal no tempo;
\item $\|DFT\|$;
\item $\angle DFT$.
\end{itemize}
\item Comparar no mesmo gr�fico a DFT e FT anal�tica
\end{enumerate}

\subsection{Casos}
$N = 512$ e resolu��o $= 100Hz$\\
$N = 2^{14}$ e resolu��o $= 20Hz$
\subsection{Sinal 2}
\begin{equation}
x(t) = 2sinc(2t) + 0.2sinc(0.2t)
\end{equation}
\begin{enumerate}[(a)]
	\item Obtenha uma express�o para $X(f)$ e repita o exemplo anterior para:
		\begin{itemize}
			\item $N= 64,\quad h = 0.5$;
			\item $N=64,\quad h = 0.05$;
			\item $N=1024,\quad h = 0.4$.
		\end{itemize}
\end{enumerate}

\end{document}
